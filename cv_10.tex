%%%%%%%%%%%%%%%%%%%%%%%%%%%%%%%%%%%%%%%%%
% Friggeri Resume/CV
% XeLaTeX Template
% Version 1.0 (5/5/13)
%
% This template has been downloaded from:
% http://www.LaTeXTemplates.com
%
% Original author:
% Adrien Friggeri (adrien@friggeri.net)
% https://github.com/afriggeri/CV
%
% License:
% CC BY-NC-SA 3.0 (http://creativecommons.org/licenses/by-nc-sa/3.0/)
%
% Important notes:
% This template needs to be compiled with XeLaTeX and the bibliography, if used,
% needs to be compiled with biber rather than bibtex.
%
%%%%%%%%%%%%%%%%%%%%%%%%%%%%%%%%%%%%%%%%%



\documentclass[]{friggeri-cv} % Add 'print' as an option into the square bracket to remove colors from this template for printing
\usepackage{enumitem}
\usepackage{fancyhdr}
\pagestyle{fancy}

\setul{0.7ex}{0.1ex}
\setulcolor{gray}

\begin{document}

\header{Marc}{ Allaire}{Product Owner | Computer Science and Networks Engineer} % Your name and current job title/field

%----------------------------------------------------------------------------------------
%	SIDEBAR SECTION
%----------------------------------------------------------------------------------------

\begin{aside} % In the aside, each new line forces a line break
\includegraphics[scale=.6]{PHOTOCV.jpg}
\section{contact}
\href{mailto:marc.allaire@live.fr}{marc.allaire@live.fr}
\href{http://www.linkedin.com/pub/marc-allaire/62/79/936}{LinkedIn://marc-allaire}
22 avenue du Général Leclerc
33110 Le Bouscat, France
+33 6 12 64 97 84
\section{Langues}
Français langue maternelle
Anglais bilingue
Espagnol 
\section{Programmation}
Java, JavaEE
Haskell, 
C, ADA, VHDL
\LaTeX, MS Word, Powerpoint
UNIX, Linux, Windows
\end{aside}

%----------------------------------------------------------------------------------------
%	EDUCATION SECTION
%----------------------------------------------------------------------------------------
\vspace{-0.2cm}

\section{Formation}


\begin{entrylist}
%------------------------------------------------
\entry
{2009--2014}
{Master {\normalfont en Ingénierie Informatique et Réseaux}}
{INSA, Toulouse}
{\emph{Majeure Réseaux et Télécommunication}\\
%\begin{itemize}
%\item Développement logiciel et conception de systèmes complexes en utilisant différents paradigmes de programmation (Object, Fonctionnelle, Impératif)
%\item Connaissance approfondie des réseaux informatiques, des systèmes distribués et communiquants, mise en application des contraintes de sécurité, de qualité de service et de temps réel
Formation Généraliste, curiosité, envie d'apprendre et facilité d'assimilation de nouveaux concepts
%\end{itemize}
\vspace{.2cm}}
%------------------------------------------------
\entry
{2012--2014}
{Master {\normalfont en Management Stratégique}}
{IAE, Toulouse}
{\emph{Majeure Stratégie des industries innovantes}\\
%\begin{itemize}
Approche multidisciplinaire de l'innovation et de la stratégie d'entreprise (droit, finance, marketing, management)
%\item Juste équilibre entre expérience pratique sur des cas concrets et connaissance théorique
%\end{itemize} 
\vspace{.2cm}}
%------------------------------------------------
\entry
{Spring 2012}
{Bachelor of Science {\normalfont en Computer Science}}
{\ul{University of Arizona, Tucson, USA}}
{\vspace{-10pt}}
%------------------------------------------------
\end{entrylist}

%----------------------------------------------------------------------------------------
%	WORK EXPERIENCE SECTION
%----------------------------------------------------------------------------------------

\section{Expérience}


\begin{entrylist}
%------------------------------------------------
\entry
{2014 -- 6 mois}
{NICTA, \textsc{Queensland Research Laboratory, Software Systems}}
{\ul{Brisbane, Australie}}
{\emph{Extension d'un outil de conformité des process d'entreprise et revision de la stratégie produit}
\begin{itemize}
\item Reprise de code et développement d'extensions dans plusieurs langages (Java, Scala, Groovy, JavaScript)
\item Travail de recherche sur l'extension du modèle et démonstration formelle de l'équivalence des théories
%\item Auto formation sur la théorie qui soutient le logiciel à partir de la litérature disponible 
\item Mise en application des modèles de stratégie et marketing pour revoir la mise sur le marché du logiciel
\end{itemize}
}

\entry
{2013 -- 4 mois}
{\textsc{Airbus Defence and Space}}
{Toulouse, France}
{\emph{Stage Développement d'une interface pour simulateur satellite}
\begin{itemize}
\item Application des principes de la programmation objet sur une application professionnelle en partant de zéro
\item Auto apprentissage de JavaFX et utilisation intensive pour le design de l'interface utilisateur
%\item Optimisation du code pour le traitement de grosses quantité de données en temps réel
\end{itemize}
}
%------------------------------------------------
\entry
{2013 -- 4 mois}
{INSA Toulouse}
{Toulouse, France}
{\emph{Développement d'un emulateur de trafic dans un ESB pour Thales}
\begin{itemize}
%\item Etude comparative des solutions de virtualisation open-source du marché
\item Développement d'application orientée service et Middleware (Web Service, BPEL, XSLT) sur GlassFish
\item Connaissance théorique approfondie des middleware et ESB pour applications d'entreprises distribuées
\end{itemize}}
%-----------------------------------------------
%\entry
%{2012 -- 4 mois}
%{INSA Toulouse}
%{Toulouse, France}
%{\emph{Projet tutoré Maquette de cloud et sondes de sécurité}
%\begin{itemize}
%\item Etude comparative des solutions de virtualisation open-source du marché
%\item Déploiement d'une maquette de cloud avec OpenStack et Xen sur 3 machines en réseau
%\item Déploiement de sondes de sécurité (Snort, OSSEC) selon différentes politiques
%\end{itemize}}
%------------------------------------------------
\entry
{3 mois}
{INSA Toulouse}
{Toulouse, France}
{\emph{Projet gestion de la QoS pour des appels SIP dans un domaine DiffServ}
\begin{itemize}
%\item Definition et implementation du réseau IP avec des routeurs Cisco et Linux
\item Mise en place de VPN L2 et L3 et gestion des ressources avec l'utilitaire Linux TC
\item Modification d'un proxy SIP open source en Java pour intégrer nos mécanismes de QoS
\end{itemize}}


\end{entrylist}

\vspace{-0.2cm}

%-----------------------------------------------------------------------------------------
%    ATOUTS MAJEURS
%-----------------------------------------------------------------------------------------
\section{Atouts Majeurs}


\textbf{\large Programmation~:}
\vspace{-0.1cm}
\begin{itemize}[noitemsep,nolistsep]
\item \textbf{Orientée Objet} : Maitrise des concepts de conception et de programmation objet, implémenté en Java
\item \textbf{Fonctionnelle} : Apprentissage du Haskell et des bases théoriques sur mon temps libre
\item \textbf{Middleware et SOA} : Développement d'applications orientés service (OpenESB, SOAP, XML, WebServices)
\item \textbf{Impérative et temps réel} : C sur microcontrôleur, programmation temps réel sous Xenomai 
% assembleur MIPS 
et VHDL sur FPGA
!%\item \textbf{Version Control} : Développement de projets avec GIT, SVN et CVS
\item \textbf{Cloud} : Déploiement d'infrastructure virtualisée, Connaissances théoriques
\item \textbf{Autres} : Connaissances en sécurité des applications, Bases de données relationnelles et SQL
\end{itemize}

\textbf{\large Réseaux~:} 
\vspace{-0.1cm}
\begin{itemize}[noitemsep,nolistsep]
\item \textbf{Locaux} : Déploiement, architecture et protocoles de réseau local filaire et sans-fil IPv4 et IPv6, sécurité des réseaux
\item \textbf{Grande distance} : Connaissance approfondie des protocoles de routage de l'internet (RIP, OSPF, IGRP, BGP)\\
Technologies et protocoles des réseaux grande distance (ATM, VPN, MPLS)
\item \textbf{Matériel} : Expérience pratique sur des routeurs Cisco et Linux, connaissances des utilitaires réseaux sur Linux
\item \textbf{Télécommunications} : Réseaux cellulaires (GSM, UMTS, 3G), Couche physique (Antennes, Modulations)
\end{itemize}

\textbf{\large Stratégie} :
\vspace{-0.1cm}
\begin{itemize}[noitemsep,nolistsep]
\item \textbf{Théorie \& Pratique} : Compétences sur des sujets stratégiques et de gestion, Mise en pratique des modèles théoriques
\item \textbf{Double compétence} : Vision globale du processus d'innovation, Recul stratégique sur le travail d'ingénieur
\end{itemize}

\textbf{\large Personnel~:} 
\vspace{-0.1cm}
\begin{itemize}[noitemsep,nolistsep]
\item \textbf{Ouverture d'esprit et volonté} : Deux projets d'étude et de travail à l'étranger (USA et Australie). Connaissance approfondie de la culture anglo-saxonne et de la langue anglaise. Habitué à sortir de ma zone de confort. Nombreux voyages
\item \textbf{Patience, motivation, persévérence et esprit d'équipe} : des qualités acquises au cours de mes projets professionnels mais aussi au travers de mes activités sportives comme le golf, la course à pied, le hockey
\item \textbf{Curiosité} : Plaisir d'apprendre et de se confronter à la nouveauté, Formation rapide à de nouveaux concepts
\end{itemize}


%----------------------------------------------------------------------------------------
%	INTERESTS SECTION
%----------------------------------------------------------------------------------------

%\section{Centres d'Intéret}

%\textbf{Voyages:} Stage de fin d'étude en Australie (Perfectionnement de l'anglais professionel et technique) et
%Voyages en Océanie (Nouvelle-Zélande, Bali, Nouvelle-Calédonie)\\
%Semestre aux USA, visite de 9 Etats (connaissance approfondie de la langue et de la culture américaine)\\
%Deux semaines d’immersion au Mexique et visite de plusieurs pays Européens (Irlande, Pays-Bas, Angleterre, Espagne)\\[5pt]
%\textbf{Musique et Sports} : Piano (10 ans) et basse (5 ans) --- Golf (index 12.0) --- Championnats de France de hockey en salle\\[5pt]
%\textbf{Technologie} : Actif sur les sites de question réponse comme stackoverflow.com et sur IRC\\


\cfoot{\scriptsize \textcopyright Marc Allaire using a template by Adrien Friggeri --- Licence Creative Commons (CC BY-NC-SA 3.0)}

\end{document}